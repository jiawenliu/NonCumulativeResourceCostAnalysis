%
%
\subsection{Syntax}
% \mg{It is ok to list all the operations in the appendix but for the main paper it is better to save space.}
\[
\begin{array}{llll}
\mbox{Arithmetic Operators} 
& \oplus_a & ::= & + ~|~ - ~|~ \times 
%
~|~ \div ~|~ \max ~|~ \min\\  
% ~|~ \div \\  
\mbox{Boolean Operators} 
& \oplus_b & ::= & \lor ~|~ \land
\\
%
\mbox{Relational Operators} 
& \sim & ::= & < ~|~ \leq ~|~ == 
\\  
%
\mbox{Arithmetic Expression} 
& \aexpr & ::= & 
n ~|~ {x} ~|~ \aexpr \oplus_a \aexpr  
 ~|~ \elog \aexpr  ~|~ \esign \aexpr
\\
%
\mbox{Boolean Expression} & \bexpr & ::= & 
%
\etrue ~|~ \efalse  ~|~ \neg \bexpr
 ~|~ \bexpr \oplus_b \bexpr
%
~|~ \aexpr \sim \aexpr 
\\
%
\mbox{Expression} & \expr & ::= & v ~|~ \aexpr ~|~ \bexpr ~|~ [\expr, \dots, \expr]
\\  
%
\mbox{Value} 
& v & ::= & { n ~|~ \etrue ~|~ \efalse ~|~ [] ~|~ [v, \dots, v]}  
\\ 
&&&
\highlight
{
~|~ (r, x_1, \ldots, x_n) := c
}
\\
%
\mbox{ Expression for \highlight{NL Quantitative Property}} 
& {\qexpr} & ::= 
& 
{ \qval ~|~ 
% \aexpr ~|~ \qexpr \oplus_a \qexpr ~|~ \chi[\aexpr]
} 
\\
%
\mbox{Value for \highlight{NL Quantitative Property}} & \qval & ::= 
& 
% {n ~|~ \chi[n] ~|~ \qval \oplus_a  \qval ~|~ n \oplus_a  \chi[n]
    % ~|~ \chi[n] \oplus_a  n}
\\
% \\%
\mbox{Label} 
& l & ::= & (n \in \mathbb{N} \cup \{\lin, \lex\}) ~|~ (l, n)
\\ 
%
\mbox{Labeled Command} 
& {c} & ::= &  
\clabel{\assign{x}{\expr}}^l 
~|~ \highlight{\clabel{\assign{x}{\nonlinear(\qexpr)}}^l}
~|~  \clabel{\eskip}^l
~|~ \ewhile \clabel{\bexpr}^{l} \edo {c}
~|~ \eif(\clabel{\bexpr}^{l} , {c}, {c}) 
\\ 
&&&
~|~ \clabel{\efun}^l: x(r, x_1, \ldots, x_n) := c
~|~ \clabel{\assign{x}{\ecall(x, e_1, \ldots, e_n)}}^l
~|~ {c};{c}  
\\ 
% \\
\mbox{Event} 
& \event & ::= & 
% ~|~ ({x}, l, v, \qval)
({x}, l, v, \bullet)   \qquad \mbox{Assignment Event} \\
&&& ~|~(\bexpr, l, v, \bullet)  \qquad \mbox{Testing Event}
\\
&&& ~|~ \highlight{({x}, l, v, \qval)  \qquad \mbox{NL Property Assignment Event}}
\\
\mbox{Trace} & \trace
& ::= & [] ~|~ \trace :: \event\\
\end{array}
\]
We use following notations to represent the set of corresponding terms:
\[
\begin{array}{lll}
\mathcal{VAR} & : & \mbox{Set of Variables}  
\\ 
%
\mathcal{VAL} & : & \mbox{Set of Values} 
\\ 
%
\mathcal{QVAL} & : & \mbox{Set of Values for Non-Linear Property} 
\\ 
%
\cdom & : & \mbox{Set of Commands} 
\\ 
%
\eventset  & : & \mbox{Set of Events}  
\\
%
\eventset^{\asn}  & : & \mbox{Set of Assignment Events}  
\\
%
\eventset^{\test}  & : & \mbox{Set of Testing Events}  
\\
%
\ldom  & : & \mbox{Set of Labels}  
\\
%%
\mathcal{VAL}  & : & \mbox{Set of Labeled Variables}  
\\
%%
% \dbdom  & : & \mbox{{Set of Databases}} 
% \\
%
{\mathcal{T}} & : & \mbox{Set of Traces}
\\
%
\mathcal{T}_0(c) & : & \mbox{Set of Initial Traces, where all the input variables of the program $c$ are initialized.
}
\end{array}
\]
%
%
%
%
%
% \subsection{Trace-based Operational Semantics for Language \mg{What is ``Language''?}}
\subsection{Trace-based Operational Semantics}
\subsubsection{Event}
vent projection operators $\pi_i$ projects the $i$th element from an event: 
\\
$\pi_i : 
\eventset \to \mathcal{VAR}\cup \mbox{Boolean Expression}  \cup \mathbb{N} \cup \mathcal{VAL} \cup \mathcal{QVAL} $ 
% \wqside{use b for Boolean expression?}
\\
% $\pi_{(i,j)} (\event) \triangleq (\pi_i(\event), \pi_j(\event)) $
% %
% \\
% Event Signature : $\pi_{\sig} : \eventset \to (\mathcal{VAR}\cup \mbox{Boolean Expression}) \times\mathbb{N}\times \mathbb{N}$
% \[
% \begin{array}{lll}
% \pi_{\sig} (x, l, n, v) \triangleq (x, l, n)
% &
% \pi_{\sig} (x, l, n, \qval, v) \triangleq (x, l, n, \query)
% &
% \pi_{\sig} (b, l, n, v)  \triangleq (b, l, n)
% \end{array}
% \]
%
%
Free Variables: $FV: \expr \to \mathcal{P}(\mathcal{VAR})$, the set of free variables in an expression.
\\
$FV(\qexpr)$ is the set of free variables in the query expression $\qexpr$.


\subsubsection{Trace}
 \begin{defn}[Trace Concatenation, $\tracecat: \mathcal{T} \to \mathcal{T} \to \mathcal{T}$]
Given two traces $\trace_1, \trace_2 \in \mathcal{T}$, the trace concatenation operator 
$\tracecat$ is defined as:
\[
  \trace_1 \tracecat \trace_2 \triangleq
  \left\{
  \begin{array}{ll} 
     \trace_1 & \trace_2 = [] \\
     (\trace_1  \tracecat \trace_2')  :: \event & \trace_2 = \trace_2' :: \event
  \end{array}
  \right.
\]
\end{defn}
%
% \todo{ need to consider the occurrence times }
% \\
% \mg{This definition is not well given. You use a different operator to define it t[:e] which you say it is a shorthand. It cannot be a shorthand because it is used in the definition. I think you need to define two operations, either in sequence or mutually recursive.}
% \jl{I moved this definition from the main paper into the appendix \ref{apdx:flowsto_event_soundness}. Because this operator is only being used in the soundness proof. And I also feel it doesn't worth to spend many lines in the main paper for defining this complex notation.}
% Subtrace: $[ : ] : \mathcal{{T} \to \eventset \to \eventset \to \mathcal{T}}$ 
% \wqside{Confusing, I can not understand the subtraction, it takes a trace, and two events, and this operator is used to subtract these two events?}
% \[
%   \trace[\event_1 : \event_2] \triangleq
%   \left\{
%   \begin{array}{ll} 
%   \trace'[\event_1: \event_2]             & \trace = \event :: \trace' \land \event \eventneq \event_1 \\
%   \event_1 :: \trace'[:\event_2]  & \trace = \event :: \trace' \land \event \eventeq \event_1 \\
%   {[]} & \trace = [] \\
%   \end{array}
%   \right.
% \]
% For any trace $\trace$ and two events $\event_1, \event_2 \in \eventset$,
% $\trace[\event_1 : \event_2]$ takes the subtrace of $\trace$ starting with $\event_1$ and ending with $\event_2$ including $\event_1$ and $\event_2$.
% \\
% We use $\trace[:\event_2] $ as the shorthand of subtrace starting from head and ending with $\event_2$, and similary for $\trace[\event_1:]$.
% \[
%   \trace[:\event] \triangleq
%   \left\{
%   \begin{array}{ll} 
%  \event' :: \trace'[: \event]             & \trace = \event' :: \trace' \land \event' \eventneq \event \\
%   \event'  & \trace = \event' :: \trace' \land \event' \eventeq \event \\
%   {[]}  & \trace = [] 
%   \end{array}
%   \right.
% % \]
% % \[
%   \quad
%   \trace[\event: ] \triangleq
%   \left\{
%   \begin{array}{ll} 
%   \trace'[\event: ]     & \trace =  \event' :: \trace' \land \event \eventneq \event' \\
%   \event' :: \trace'  & \trace = \event' :: \trace' \land \event \eventeq \event' \\
%   {[ ] } & \trace = []
%   \end{array}
%   \right.
% \]
% %
% \mg{why in the next definition you use ( ) while in the previous ones you didn't? They seem like the same cases. And why you use o.w. instead of []?}
% An event $\event \in \eventset$ belongs to a trace $\trace$, i.e., $\event \eventin \trace$ are defined as follows:
% %
% \begin{equation}
%   \event \eventin \trace  
%   \triangleq \left\{
%   \begin{array}{ll} 
%     \etrue                  & \trace =  (\event' :: \trace') \land (\event \eventeq \event')
%                               \\
%     \event \eventin \trace' & \trace =  (\event' :: \trace') \land (\event \eventneq \event') \\ 
%     \efalse                 & o.w.
%   \end{array}
%   \right.
% \end{equation}
\begin{defn}(An Event Belongs to A Trace)
  An event $\event \in \eventset$ belongs to a trace $\trace$, i.e., $\event \in \trace$ are defined as follows:
%
\begin{equation}
  \event \in \trace  
  \triangleq \left\{
  \begin{array}{ll} 
    \etrue                  & \trace =  \trace' :: \event'
     \land \event = \event'
                              \\
    \event \in \trace' & \trace =  \trace' :: \event'
    \land \event \neq \event' \\ 
    \efalse                 & \trace = []
  \end{array}
  \right.
\end{equation}
As usual, we denote by $\event \notin \trace$ that the event $\event$ doesn't belong to the trace $\trace$.
\end{defn}
%
Counting operator $\vcounter : \mathcal{T} \to \mathbb{N} \to \mathbb{N}$ whose behavior is defined as follows,
% \[
% \begin{array}{lll}
% \vcounter(\trace :: (x, l, v, \bullet) ) l \triangleq \vcounter(\trace) l + 1
% &
% \vcounter(\trace  ::(b, l, v, \bullet) ) l \triangleq \vcounter(\trace) l + 1
% &
% \vcounter(\trace  :: (x, l, v, \qval) ) l \triangleq \vcounter(\trace) l + 1
% \\
% \vcounter(\trace  :: (x, l', v, \bullet) ) l \triangleq \vcounter(\trace ) l, l' \neq l
% &
% \vcounter(\trace  :: (b, l', v, \bullet) ) l \triangleq \vcounter(\trace ) l, l' \neq l
% &
% \vcounter(\trace  :: (x, l', v, \qval)) l \triangleq \vcounter(\trace ) l, l' \neq l
% \\
% \vcounter({[]}) l \triangleq 0
% &&
% \end{array}
% \]
\[
\begin{array}{lll}
\vcounter(\trace :: (x, l, v, \bullet), l ) \triangleq \vcounter(\trace, l) + 1
&
\vcounter(\trace  ::(b, l, v, \bullet), l) \triangleq \vcounter(\trace, l) + 1
&
\vcounter(\trace  :: (x, l, v, \qval), l) \triangleq \vcounter(\trace, l) + 1
\\
\vcounter(\trace  :: (x, l', v, \bullet), l) \triangleq \vcounter(\trace, l), l' \neq l
&
\vcounter(\trace  :: (b, l', v, \bullet), l) \triangleq \vcounter(\trace, l), l' \neq l
&
\vcounter(\trace  :: (x, l', v, \qval), l) \triangleq \vcounter(\trace, l), l' \neq l
\\
\vcounter({[]}, l) \triangleq 0
&&
\end{array}
\]
%
% The Latest Label $\llabel : \mathcal{T} \to \mathcal{VAR} \to \mathbb{N}$ 
% The label of the latest assignment event which assigns value to variable $x$.
% \[
%   \begin{array}{lll}
% \llabel((x, l, v):: \trace) x \triangleq l
% &
% \llabel((b, l, v)):: \trace x \triangleq \llabel(\trace) x
% &
% \llabel((x, l, \qval, v):: \trace) x \triangleq l
% \\
% \llabel((y, l, v):: \trace) x \triangleq \llabel(\trace ) x
% &
% \llabel((y, l, \qval, v):: \trace) x \triangleq \llabel(\trace ) x
% \\
% \llabel({[]}) x \triangleq \bot
% &&
% \end{array}
% \]
%
% \todo{wording}
% \mg{This wording needs to be fixed. Also notice that the type is wrong, a label is not always returned.}
%  The Latest Label $\llabel : \mathcal{T} \to \mathcal{VAR} \to \mathbb{N}$ 
% The label of the latest assignment event which assigns value to variable $x$.
% \[
%   \begin{array}{lll}
% \llabel(\trace  \tracecat [(x, l, v)]) x \triangleq l
% &
% \llabel(\trace  \tracecat [(b, l, v)]) x \triangleq \llabel(\trace) x
% &
% \llabel(\trace  \tracecat [(x, l, \qval, v)]) x \triangleq l
% \\
% \llabel(\trace  \tracecat [(y, l, v)]) x \triangleq \llabel(\trace ) x
% &
% \llabel(\trace  \tracecat [(y, l, \qval, v)]) x \triangleq \llabel(\trace ) x
% \\
% \llabel({[]}) x \triangleq \bot
% &&
% \end{array}
% \]
We introduce an operator $\llabel : \mathcal{T} \to \mathcal{VAR} \to \ldom \cup \{\bot\}$, which 
takes a trace and a variable and returns the label of the latest assignment event which assigns value to that variable.
Its behavior is defined as follows,
% \begin{defn}[Latest Label]
  \[
    % \begin{array}{lll}
  \llabel(\trace  :: (x, l, \_, \_)) x \triangleq l
  ~~~
  \llabel(\trace  :: (y, l, \_, \_)) x \triangleq \llabel(\trace ) x, y \neq x
  % &
  ~~~
  \llabel(\trace :: (b, l, v, \bullet)) x \triangleq \llabel(\trace) x
  % &
  % \\
  % \llabel(\trace  :: (y, l, v, \bullet)) x \triangleq \llabel(\trace ) x
  % &
  % \llabel(\trace :: (y, l, v, \qval)) x \triangleq \llabel(\trace ) x
  % &
  ~~~
  \llabel({[]}) x \triangleq \bot
  % \end{array}
  \]
% \end{defn}
%
% \mg{This wording needs to be fixed but also the description does not make sense. This operator seems to just collect all the labels in a trace. Again, this definition would be shorter with a more uniform definition of events.}
% The Trace Label Set $\tlabel : \mathcal{T} \to \mathcal{P}{(\mathbb{N})}$ 
% The label of the latest assignment event which assigns value to variable $x$.
% \[
%   \begin{array}{llll}
% \tlabel_{(\trace  \tracecat [(x, l, v)])} \triangleq \{l\} \cup \tlabel_{(\trace )}
% &
% \tlabel_{(\trace  \tracecat [(b, l, v)])} \triangleq \{l\} \cup \tlabel_{(\trace)}
% &
% \tlabel_{(\trace  \tracecat [(x, l, \qval, v)])} \triangleq \{l\} \cup \tlabel_{(\trace)}
% &
% \tlabel_{[]} \triangleq \{\}
% \end{array}
% \]
% \begin{defn}
  The operator $\tlabel : \mathcal{T} \to \mathcal{P}{(\ldom)}$ gives the set of labels in every event belonging to 
  a trace, whoes behavior is defined as follows,
\[
  % \begin{array}{llll}
\tlabel{(\trace  :: (\_, l, \_, \_)])} \triangleq \{l\} \cup \tlabel{(\trace )}
~~~
\tlabel({[ ]}) \triangleq \{\}
% \end{array}
\]
% \end{defn}%
% Given a trace $\trace$, its -processed trace $\trace$ is computed by a function $p : \trace \to \trace$ as follows:
% \[
%   \trace \triangleq
%   \left\{
%   \begin{array}{ll} 
%   p(\trace' \tracecate (x, l, v)) & = p(\trace') \tracecate (x, l, \vcounter(\trace') l + 1, v) \\
%   p(\trace' \tracecate (b, l, v)) & = p(\trace') \tracecate (b, l, \vcounter(\trace') l + 1, v) \\
%   p(\trace' \tracecate (x, l, \qval, v)) & = p(\trace') \tracecate (x, l, \vcounter(\trace') l + 1, \qval, v) \\
%   p([]) & = []
%   \end{array}
%   \right.
% \]
%
%
% $\mathcal{T}$ : Set of Well-formed Traces (in Definition~\ref{def:wf_trace})
%
%
% \\
%

%
% \mg{The next lemma is trivial but it still needs a proof sketch.}
If we observe the operational semantics rules, we can find that no rule will shrink the trace. 
So we have the Lemma~\ref{lem:tracenondec} with proof in Appendix~\ref{apdx:lemma_sec123}, 
specifically the trace has the property that its length never decreases during the program execution.
\begin{lem}
[Trace Non-Decreasing]
\label{lem:tracenondec}
For every program $c \in \cdom$ and traces $\trace, \trace' \in \mathcal{T}$, if 
$\config{c, \trace} \rightarrow^{*} \config{\eskip, \trace'}$,
then there exists a trace $\trace'' \in \mathcal{T}$ with $\trace \tracecat \trace'' = \trace'$
%
$$
\forall \trace, \trace' \in \mathcal{T}, c \st
\config{c, \trace} \rightarrow^{*} \config{\eskip, \trace'} 
\implies \exists \trace'' \in \mathcal{T} \st \trace \tracecat \trace'' = \trace'
$$
\end{lem}
% \begin{proof}
%   Taking arbitrary trace $\trace \in \mathcal{T}$, by induction on program $c$, we have the following cases:
%   \caseL{$c = [\assign{x}{\expr}]^{l}$}
%   By the evaluation rule $\rname{assn}$, we have
%   $
%   {
%   \config{[\assign{{x}}{\aexpr}]^{l},  \trace } 
%   \xrightarrow{} 
%   \config{\eskip, \trace :: ({x}, l, v, \bullet)}
%   }$, for some $v \in \mathbb{N}$.
%   \\
%   Picking $\trace' = \trace ::({x}, l, v, \bullet)$ and $\trace'' =  [({x}, l, v, \bullet) ]$,
%   it is obvious that $\trace \tracecat \trace'' = \trace'$.
%   % \\
%   % There are 2 cases, where $l' = l$ and $l' \neq l$.
%   % \\
%   % In case of $l' \neq l$, we know $\event \not\eventin \trace$, then this Lemma is vacuously true.
%   %   \\
%   %   In case of $l' = l$, by the abstract Execution Trace computation, we know 
%   %   $\absflow(c) = \absflow'([x := \expr]^{l}; \clabel{\eskip}^{l_e}) = \{(l, \absexpr(\expr), l_e)\}$  
%   %   \\
%   % Then we have $\absevent = (l, \absexpr(\expr), l_e) $ and $\absevent \in \absflow(c)$.
%   \\
%   This case is proved.
%   \caseL{$c = [\assign{x}{\query(\qexpr)}]^{l'}$}
%   This case is proved in the same way as \textbf{case: $c = [\assign{x}{\expr}]^l$}.
%   \caseL{$\ewhile [b]^{l_w} \edo c$}
%   By the first rule applied to $c$, there are two cases:
%   \subcaseL{$\textbf{while-t}$}
%   If the first rule applied to is $\rname{while-t}$, we have
%   \\
%   $\config{{\ewhile [b]^{l_w} \edo c_w, \trace}}
%     \xrightarrow{} 
%     \config{{
%     c_w; \ewhile [b]^{l_w} \edo c_w,
%     \trace :: (b, l_w, \etrue, \bullet)}}~ (1)
%   $.
%   \\
%   Let $\trace_w' \in \mathcal{T}$ be the trace satisfying following execution:
%   \\
%   $
%   \config{{
%   c_w,
%   \trace :: (b, l_w, \etrue, \bullet)}}
%   \xrightarrow{*} 
%   \config{{
%   \eskip, \trace_w'}}
% $
% \\
% By induction hypothesis on sub program $c_w$ with starting trace 
% $\trace :: (b, l_w, \etrue, \bullet)$ and ending trace $\trace_w'$, 
% we know there exist
% $\trace_w \in \mathcal{T}$ such that $\trace_w' = \trace :: (b, l_w, \etrue, \bullet) \tracecat \trace_w$.
% \\
% Then we have the following execution continued from $(1)$:
% \\
% $
% \config{{\ewhile [b]^{l_w} \edo c_w, \trace}}
%     \xrightarrow{} 
%     \config{{
%     c_w; \ewhile [b]^{l_w} \edo c_w,
%     \trace :: (b, l_w, \etrue, \bullet)}}
%     \xrightarrow{*} 
%     \config{\ewhile [b]^{l_w} \edo c_w, \trace :: (b, l_w, \etrue, \bullet) \tracecat \trace_w}
%     ~(2)
% $
% By repeating the execution (1) and (2) until the program is evaluated into $\eskip$,
% with trace $\trace_w^{i'} $ for $i = 1, \cdots, n n \geq 1$ in each iteration, we know 
% in the $i-th$ iteration,
%  there exists  $\trace_w^i \in \mathcal{T}$ such that  
% $\trace_w^{i'} = \trace_w^{(i-1)'} :: (b, l_w, \etrue, \bullet) ++ \trace_w^{i'}$
% \\
% Then we have the following execution:
% \\
% $
% \config{{\ewhile [b]^{l_w} \edo c_w, \trace}}
%     \xrightarrow{} 
%     \config{{
%     c_w; \ewhile [b]^{l_w} \edo c_w,
%     \trace :: (b, l_w, \etrue, \bullet)}}
%     \xrightarrow{*} 
%     \config{\ewhile [b]^{l_w} \edo c_w, \trace_w^{n'}}
%     \xrightarrow{}^\rname{{while-f}}
%     \config{\eskip, \trace_w^{n'}:: (b, l_w, \efalse, \bullet)}
% $ and $\trace_w^{n'} = \trace :: (b, l_w, \etrue, \bullet) \tracecat \trace_w^{1} :: \cdots :: (b, l_w, \etrue, \bullet) \tracecat \trace_w^{n} $.
% \\
% Picking $\trace' = \trace_w^{n'} :: (b, l_w, \efalse, \bullet)$ and $\trace'' = [(b, l_w, \etrue, \bullet)] \tracecat \trace_w^{1} :: \cdots :: (b, l_w, \etrue, \bullet) \tracecat \trace_w^{n}$,
% we have 
% $\trace ++ \trace'' = \trace'$.
% \\
% This case is proved.
%   \subcaseL{$\textbf{while-f}$}
%   If the first rule applied to $c$ is $\rname{while-f}$, we have
%   \\
%   $
%   {
%     \config{{\ewhile [b]^{l_w} \edo c_w, \trace}}
%     \xrightarrow{}^\rname{while-f}
%     \config{{
%     \eskip,
%     \trace :: (b, l_w, \efalse, \bullet)}}
%   }$,
%   In this case, picking $\trace' = \trace ::(b, l_w, \efalse, \bullet)$ and $\trace'' =  [(b, l_w, \efalse, \bullet) ]$,
%   it is obvious that $\trace \tracecat \trace'' = \trace'$.
%   \\
%   This case is proved.
%   \caseL{$\eif([b]^l, c_t, c_f)$}
%   This case is proved in the same way as \textbf{case: $c = \ewhile [b]^{l} \edo c$}.
%   \caseL{$c = c_{s1};c_{s2}$}
%  By the induction hypothesis on $c_{s1}$ and $c_{s2}$ separately,
%  we have this case proved.
% \end{proof}
%
% \todo{more explanation}
% \mg{This corollary needs some explanation. In particular, we should stress that $\event$ and $\event'$ may differ in the query value.}
Since the equivalence over two events is defined over the query value equivalence, 
when there is an event belonging to a trace, 
if this event is a query assignment event, 
it is possible that 
the event showing up in this trace has a different form of query value, 
but they are equivalent by Definition~\ref{def:query_equal}.
So we have the following Corollary~\ref{coro:aqintrace} with proof in Appendix~\ref{apdx:lemma_sec123}.
\begin{coro}
\label{coro:aqintrace}
For every event and a trace $\trace \in \mathcal{T}$,
if $\event \in \trace$, 
then there exist another event $\event' \in \eventset$ and traces $\trace_1, \trace_2 \in \mathcal{T}$
such that $\trace_1 \tracecat [\event'] \tracecat \trace_2 = \trace $
with 
$\event$ and $\event'$ equivalent but may differ in their query value.
\[
  \forall \event \in \eventset, \trace \in \mathcal{T} \st
\event \in \trace \implies \exists \trace_1, \trace_2 \in \mathcal{T}, 
\event' \in \eventset \st (\event \in \event') \land \trace_1 \tracecat [\event'] \tracecat \trace_2 = \trace  
\]
\end{coro}
\subsubsection{Environment}
Environment $ \env : {\mathcal{T}}  \to \mathcal{VAR} \to \mathcal{VAL} \cup \{\bot\}$
\[
\begin{array}{lll}
\env(\trace  \traceadd (x, l, v, \bullet)) x \triangleq v
&
\env(\trace \traceadd (y, l, v, \bullet)) x \triangleq \env(\trace) x, y \neq x
&
\env(\trace \traceadd (b, l, v, \bullet)) x \triangleq \env(\trace) x
\\
\env(\trace \traceadd (x, l, v, \qval)) x \triangleq v
&
\env(\trace \traceadd (y, l, v, \qval)) x \triangleq \env(\trace) x, y \neq x
&
\env({[]} ) x \triangleq \bot
\end{array}
\]

\subsubsection{Operational Semantics Rules}

{
\begin{mathpar}
\boxed{ \config{\trace,\aexpr} \aarrow v \, : \, \mbox{Trace  $\times$ Arithmetic Expr $\Rightarrow$ Arithmetic Value} }
\\
% \text{\mg{Missing. Without these rules it is difficult to understand why we need a trace to evaluate expressions.}}
% \\
\inferrule{ 
  \empty
}{
 \config{\trace,  n} 
 \aarrow n
}
\and
\inferrule{ 
  \env(\trace) x = v
}{
 \config{\trace,  x} 
 \aarrow v
}
\and
\inferrule{ 
  \config{\trace, \aexpr_1} \aarrow v_1
  \and 
  \config{\trace, \aexpr_2} \aarrow v_2
  \and 
   v_1 \oplus_a v_2 = v
}{
 \config{\trace,  \aexpr_1 \oplus_a \aexpr_2} 
 \aarrow v
}
\and
\inferrule{ 
  \config{\trace, \aexpr} \aarrow v'
  \and 
  \elog v' = v
}{
 \config{\trace,  \elog \aexpr} 
 \aarrow v
}
\and
\inferrule{ 
  \config{\trace, \aexpr} \aarrow v'
  \and 
  \esign v' = v
}{
 \config{\trace,  \esign \aexpr} 
 \aarrow v
}
\\
\boxed{ \config{\trace, \bexpr} \barrow v \, : \, \mbox{Trace $\times$ Boolean Expr $\Rightarrow$ Boolean Value} }
\\% \\
\inferrule{ 
  \empty
}{
 \config{\trace,  \efalse} 
 \barrow \efalse
}
\and 
\inferrule{ 
  \empty
}{
 \config{\trace,  \etrue} 
 \barrow \etrue
}
\and 
\inferrule{ 
  \config{\trace, \bexpr} \barrow v'
  \and 
  \neg v' = v
}{
 \config{\trace,  \neg \bexpr} 
 \barrow v
}
\and 
\inferrule{ 
  \config{\trace, \bexpr_1} \barrow v_1
  \and 
  \config{\trace, \bexpr_2} \barrow v_2
  \and 
   v_1 \oplus_b v_2 = v
}{
 \config{\trace,  \bexpr_1 \oplus_b \bexpr_2} 
 \barrow v
}
\and 
\inferrule{ 
  \config{\trace, \aexpr_1} \aarrow v_1
  \and 
  \config{\trace, \aexpr_2} \aarrow v_2
  \and 
   v_1 \sim v_2 = v
}{
 \config{\trace,  \aexpr_1 \sim \aexpr_2} 
 \barrow v
}
\\
\boxed{ \config{\trace, \expr} \earrow v \, : \, \mbox{Trace $\times$ Expression $\Rightarrow$ Value} }
\\
\inferrule{ 
  \config{\trace, \aexpr} \aarrow v
}{
 \config{\trace,  \aexpr} 
 \earrow v
}
\and
\inferrule{ 
  \config{\trace, \bexpr} \barrow v
}{
 \config{\trace,  \bexpr} 
 \earrow v
}
\and
\inferrule{ 
  \config{\trace, \expr_1} \earrow v_1
  \cdots
  \config{\trace, \expr_n} \earrow v_n
}{
 \config{\trace,  [\expr_1, \cdots, \expr_n]} 
 \earrow [v_1, \cdots, v_n]
}
\and
\inferrule{ 
  \empty
}{
 \config{\trace,  v} 
 \earrow v
}
% \\
% \boxed{ \config{\trace, \qexpr} \qarrow \qval \, : \, \mbox{Trace  $\times$ Query Expr $\Rightarrow$ Query Value} }
% \\
% \inferrule{ 
%   \config{\trace, \aexpr} \aarrow n
% }{
%  \config{\trace,  \aexpr} 
%  \qarrow n
% }
% \and
% \inferrule{ 
%   \config{\trace, \qexpr_1} \qarrow \qval_1
%   \and
%   \config{\trace, \qexpr_2} \qarrow \qval_2
% }{
%  \config{\trace,  \qexpr_1 \oplus_a \qexpr_2} 
%  \qarrow \qval_1 \oplus_a \qval_2
% }
% \and
% \inferrule{ 
%   \config{\trace, \aexpr} \aarrow n
% }{
%  \config{\trace, \chi[\aexpr]} \qarrow \chi[n]
% }
% \and
% \inferrule{ 
%   \empty
% }{
%  \config{\trace,  \qval} 
%  \qarrow \qval
% }
 \end{mathpar}
%
The trace based operational semantics rules are defined in Figure \ref{fig:os}.
%
\begin{figure}
%   \text{\mg{Several skip are missing labels. Do we need fresh labels or we reuse l?}}
%   \\
%   \text{\jl{Both are good for OS, but generate fresh label will need extra arguments in soundness proof, so rescuing l is better}}
%   \\
% \text{\mg{Also, why we use ++, cannot we just define lists as adding elements on the right?}}  \\
% \text{\jl{I was too sticky to the convention, it is a good idea to append to the left and just use $::$ as construtor}}  \\
% \text{\mg{It is also unclear why we store the boolean expression in if and while, besides the boolean value.}}\\
% \text{\jl{When proving the soundness of dependency between trace-based and program-based,}}\\
% \text{\jl{The variable used in the boolean expression is useful in proving the inversion Lemmas.}}
{
\begin{mathpar}
\boxed{
\mbox{Command $\times$ Trace}
\xrightarrow{}
\mbox{Command $\times$ Trace}
}
\and
\boxed{\config{{c, \trace}}
\xrightarrow{} 
\config{{c',  \trace'}}
}
\\
\inferrule
{
\empty
}
{
\config{\clabel{\eskip}^l,  \trace } 
\xrightarrow{} 
\config{\clabel{\eskip}^l, \trace}
}
~\textbf{skip}
%
\and
%
\inferrule
{
\event = ({x}, l, v, \bullet)
}
{
\config{[\assign{{x}}{\aexpr}]^{l},  \trace } 
\xrightarrow{} 
\config{\clabel{\eskip}^l, \trace \traceadd \event}
}
~\textbf{assn}
%
\and
%
\highlight
{
\inferrule
{
 \config{\trace, \qexpr }\qarrow \qval
 \and 
\query(\qval) = v
\and 
\event = ({x}, l, v, \qval)
}
{
\config{{[\assign{x}{\query(\qexpr)}]^l, \trace}}
\xrightarrow{} 
\config{{\clabel{\eskip}^l,  \trace \traceadd \event} }
}
~\textbf{NL}
}
%
\and
%
\inferrule
{
  \config{\trace, b} \barrow \etrue
 \and 
 \event = (b, l, \etrue, \bullet)
}
{
\config{{\ewhile [b]^{l} \edo c, \trace}}
\xrightarrow{} 
\config{{
c; \ewhile [b]^{l} \edo c,
\trace \traceadd \event}}
}
~\textbf{while-t}
%
%
\and
%
\inferrule
{
  \config{\trace, b} \barrow \efalse
 \and 
 \event = (b, l, \efalse, \bullet)
}
{
\config{{\ewhile [b]^{l}, \edo c, \trace}}
\xrightarrow{} 
\config{{
  \clabel{\eskip}^l,
\trace \traceadd \event}}
}
~\textbf{while-f}
%
%
\and
%
%
\inferrule
{
\config{{c_1, \trace}}
\xrightarrow{}
\config{{c_1',  \trace'}}
}
{
\config{{c_1; c_2, \trace}} 
\xrightarrow{} 
\config{{c_1'; c_2, \trace'}}
}
~\textbf{seq1}
%
\and
%
\inferrule
{
  \config{{c_2, \trace}}
  \xrightarrow{}
  \config{{c_2',  \trace'}}
}
{
\config{{\clabel{\eskip}^l; c_2, \trace}} \xrightarrow{} \config{{ c_2', \trace'}}
}
~\textbf{seq2}
%
\and
%
%
\inferrule
{
  \config{\trace, b} \barrow \etrue
 \and 
 \event = (b, l, \etrue, \bullet)
}
{
 \config{{
\eif([b]^{l}, c_1, c_2), 
\trace}}
\xrightarrow{} 
\config{{c_1, \trace \traceadd \event}}
}
~\textbf{if-t}
%
\and
%
\inferrule
{
 \config{\trace, b} \barrow \efalse
 \and 
 \event = (b, l, \efalse, \bullet)
}
{
\config{{\eif([b]^{l}, c_1, c_2), \trace}}
\xrightarrow{} 
\config{{c_2, \trace \traceadd \event}}
}
~\textbf{if-f}
% %
\and
%
% \highlight
{
\inferrule
{
 c' = (c)^{+n}
 \and 
 \event = (f, l, (r, x_1, \ldots, x_n) := c', \bullet)
}
{
\config{{
  [\efun]^l: f(r, x_1, \ldots, x_n) := c, \trace}}
\xrightarrow{} 
\config{{\clabel{\eskip}^l, \trace \traceadd \event}}
}
~\textbf{fun-def}
%
}
\\
% \highlight
{
%
\inferrule
{
  \config{ \trace, f} \earrow (r, x_1, \ldots, x_n) := c
\and 
\config{{
  \clabel{\assign{x_1}{e_1}}^{(l, 1)}; \ldots;
  \clabel{\assign{x_n}{e_n}}^{(l, n)}, \trace}} 
  \xrightarrow{}^* 
  \config{{\clabel{\eskip}^{(l, n)}, \trace_1}}
  \\ 
  \config{{\clabel{c}^{(l)}, \trace_1}}
  \xrightarrow{}^* 
  \config{{\clabel{\eskip}^{l}, \trace'}}
  \and
  \config{\trace', r } \earrow v
  \and
 \event = (x, l, v, \bullet)
}
{
\config{{
  \clabel{\assign{x}{\ecall(f, e_1, \ldots, e_n)}}^l, \trace}}
\xrightarrow{} 
\config{{\clabel{\eskip}^l, \trace' :: \event}}
}
~\textbf{fun-call}
}
%
\end{mathpar}
}
% \end{subfigure}
    \caption{Trace-based Operational Semantics for Language.}
    \label{fig:os}
\end{figure}
%
%
  \begin{defn}[Label Increase]
    \label{def:label_inc}  
    Label Increase $ + : {\ldom \to \mathbb{N} \to \ldom}$, increase a label $l$ by a natural number $n$:
\[
    n + n' \triangleq n'' ~ n, n' \in \mathbb{N} \land \config{[], n + n'} \aarrow n''
   \qquad (l, n) + n' \triangleq (l + n', n'') ~ n, n' \in \mathbb{N} \land \config{[], n + n'} \aarrow n''
   \]
\end{defn}
The case of $(l, n) + n'$ will never happen during evaluation.
By Operational semantics, the only place the label increase is in rule \textbf{fun-def},
$c' = (c)^{+n}$, where $c$ is the function body.
By the rule \textbf{fun-call}, and the label augment in Definition~\ref{def:comlabel_aug}, the function body $c$ will never be augmented.
%
\begin{defn}[Command Label Increase] 
  \label{def:comlabel_inc}
Command Label Increase $ {(\cdot)}{}^{+n} : {\cdom \to \cdom}$, increase the label in command by $n$.
\[
\begin{array}{ll}
  (\clabel{\assign{x}{\expr}}^l){}^{+n} & \triangleq \clabel{\assign{x}{\expr}}^{l + n}\\
(\clabel{\assign{x}{\query(\qexpr)}}^l)^{+n} & \triangleq \clabel{\assign{x}{\query(\qexpr)}}^{l + n}\\
(\clabel{\eskip}^l)^{+n} & \triangleq \clabel{\eskip}^{l + n}\\
(\ewhile \clabel{\bexpr}^{l} \edo {c'})^{+n} & \triangleq \ewhile \clabel{\bexpr}^{l+n} \edo {(c')^{+n}}\\
(\eif(\clabel{\bexpr}^{l} , {c_1}, {c_2}))^{+n}  & \triangleq \eif(\clabel{\bexpr}^{l+n} , {(c_1)^{+n}}, {(c_2)^{+n}})\\
% (\clabel{\efun}^l: x(r^l, x_1, \ldots, x_n) := c)^{+n} & \triangleq \clabel{\efun}^{l + n}: x(r^l, x_1, \ldots, x_n) := (c)^{+n} \\
(\clabel{\efun}^l: x(r^l, x_1, \ldots, x_n) := c)^{+n} & \triangleq \clabel{\efun}^{l + n}: x(r^l, x_1, \ldots, x_n) := c \\
(\clabel{\assign{x}{\ecall(x, e_1, \ldots, e_n)}}^l)^{+n} & \triangleq \clabel{\assign{x}{\ecall(x, e_1, \ldots, e_n)}}^{l + n}\\
({c_1};{c_2})^{+n} &  \triangleq {(c_1)}^{+n};{(c_2)}^{+n}
\end{array}
\]
\end{defn}
%
\begin{defn}[Command Label Augment] 
  \label{def:comlabel_aug}
  Command Label Augment $ \clabel{\cdot}^{l} : {\cdom \to \cdom}$, augment the label in command with a label $l$ 
in order to record the calling site.
\[
\begin{array}{ll}
  \clabel{\clabel{\assign{x}{\expr}}^{l'}}{}^{l} & \triangleq \clabel{\assign{x}{\expr}}^{(l, l')}\\
  \clabel{\clabel{\assign{x}{\query(\qexpr)}}^{l'}}^{l} & \triangleq \clabel{\assign{x}{\query(\qexpr)}}^{(l, l')}\\
  \clabel{\clabel{\eskip}^{l'}}^{l} & \triangleq \clabel{\eskip}^{(l, l')}\\
  \clabel{\ewhile \clabel{\bexpr}^{l'} \edo {c'}}^{l} & \triangleq \ewhile \clabel{\bexpr}^{(l, l')} \edo {(c')^{l}}\\
  \clabel{\eif(\clabel{\bexpr}^{l'} , {c_1}, {c_2})}^{l}  & \triangleq \eif(\clabel{\bexpr}^{(l, l')} , {(c_1)^{l}}, {(c_2)^{l}})\\
  \clabel{\clabel{\efun}^{l'}: x(r^l, x_1, \ldots, x_n) := c}^{l} & \triangleq \clabel{\efun}^{(l, l')}: x(r^l, x_1, \ldots, x_n) := c \\
  \clabel{\clabel{\assign{x}{\ecall(x, e_1, \ldots, e_n)}}^{l'}}^{l} & \triangleq \clabel{\assign{x}{\ecall(x, e_1, \ldots, e_n)}}^{(l, l')}\\
  \clabel{{c_1};{c_2}}^{l} &  \triangleq \clabel{c_1}^{l};\clabel{c_2}^{l}
\end{array}
\]
\end{defn}
% Each command is now labeled with a label $l$, either a natural number standing for the line of code where the command appears, or a symbol of $in$ or $ex$ used for annotating the input variables, and the exist point of the program. Notice that we associate the label $l$ to the conditional predicate $\bexpr$ in the if statement, and to the while guard counter $\bexpr$ in the $\ewhile$ statement.
% We abuse the same notation $c$ for labeled command in the rest of the paper.
% \\
% \todo{notation}
The labeled variables and assigned variables are set of variables annotated by a label. 
We use  
%$\mathcal{LVAR} = \mathcal{VAR} \times \mathcal{L} $ 
$\mathcal{LV}$ represents the universe of all the labeled variables and 
$\avar_c \in \mathcal{P}(\mathcal{VAR} \times \mathbb{N}) \subset \mathcal{LV}$ and 
$\lvar_c \in \mathcal{P}(\mathcal{VAR} \times \mathcal{L}) \subseteq \mathcal{LV}$,
represents the the set of assigned variables and labeled variables for a labeled command $c$,
defined in Definition~\ref{def:lvar} and \ref{def:avar}.
%
\\
$FV: \expr \to \mathcal{P}(\mathcal{VAR})$, computes the set of free variables in an expression. To be precise,
$FV(\aexpr)$, $FV(\bexpr)$ and $FV(\qexpr)$ represent the set of free variables in arithmetic
expression $\aexpr$, boolean expression $\bexpr$ and query expression $\qexpr$ respectively.
Labeled variables in $c$ is the set of assigned variables and all the free variables
showing up in $c$ with a default label $in$. 
The free variables
showing up in $c$, which aren't defined before be used, are actually the input variables of this program.
%
\begin{defn}[Assigned Variables (
% $\avar_{c} \subseteq \mathcal{VAR} \times \mathbb{N}$ or 
$\avar : \cdom \to \mathcal{P}(\mathcal{VAR} \times \mathbb{N})$)]
% labelled Variables 
% (
% % $\lvar_{c} \subseteq \mathcal{VAR} \times \mathbb{N}$ or 
% $\lvar : \cdom \to \mathcal{P}(\mathcal{VAR} \times \mathcal{L})$
\label{def:avar}
{
$$ \avar_{c} \triangleq
  \left\{
  \begin{array}{ll}
      \{{x}^l\}                   
      & {c} = [{\assign x e}]^{l} 
      \\
      \{{x}^l\}                   
      & {c} = [{\assign x \query(\qexpr)}]^{l} 
      \\
      \avar_{{c_1}} \cup \avar_{{c_2}}  
      & {c} = {c_1};{c_2}
      \\
      \avar_{{c}} \cup \avar_{{c_2}} 
      & {c} =\eif([\bexpr]^{l}, c_1, c_2) 
      \\
      \avar_{{c}'}
      & {c}   = \ewhile ([\bexpr]^{l}, {c}')
\end{array}
\right.
$$
}
\end{defn}
%

\begin{defn}[labelled Variables 
(
% $\lvar_{c} \subseteq \mathcal{VAR} \times \mathbb{N}$ or 
$\lvar : \cdom \to \mathcal{P}(\mathcal{LV})$]
\label{def:lvar}
{
$$
  \lvar_{c} \triangleq
  \left\{
  \begin{array}{ll}
      \{{x}^l\} \cup FV(\expr)^{in}                  
      & {c} = [{\assign x e}]^{l} 
      \\
      \{{x}^l\}   \cup FV(\qexpr)^{in}                
      & {c} = [{\assign x \query(\qexpr)}]^{l} 
      \\
      \lvar_{{c_1}} \cup \lvar_{{c_2}}  
      & {c} = {c_1};{c_2}
      \\
      \lvar_{{c}} \cup \lvar_{{c_2}} \cup FV(\bexpr)^{in}
      & {c} =\eif([\bexpr]^{l}, c_1, c_2) 
      \\
      \lvar_{{c}'} \cup FV(\bexpr)^{in}
      & {c}   = \ewhile ([\bexpr]^{l}, {c}')
\end{array}
\right.
$$
}
\end{defn}
%
%
%
Every labeled variable in a program is unique, formally as follows with proof in Appendix~\ref{apdx:lemma_sec123}.
\begin{lem}[Uniqueness of the Labeled Variables]
  \label{lem:lvar_unique}
  For every program $c \in \cdom$ and every two labeled variables such that
  $x^i, y^j \in \lvar(c)$, then $x^i \neq y^j$.
  \[
    \forall c \in \cdom, x^i, y^j \in \mathcal{L} \st x^i, y^j \in \lvar(c)\implies x^i \neq y^j.
    \]
\end{lem}
%
\begin{defn}[Non-Linear Property Variables ($\qvar: \cdom \to \mathcal{P}(\mathcal{LV})$)] 
  \label{def:qvar}
Given a program $c$, its query variables 
% \mg{it seems you are missing the $_c$ subscript. Also, this is a minor point but I don't think it is a good idea to use a subscript, cannot you just use $\qvar(c)$.}
$\qvar(c)$ is the set of variables set to the result of a query in the program.
% \jl{fixed}
It is defined as follows:
{
$$
  % \qvar_{{c}} \triangleq
  \qvar(c) \triangleq
  \left\{
  \begin{array}{ll}
      \{\}                  
      & {c} = [{\assign x \expr}]^{l} 
      \\
      \{{x}^l\}                  
      & {c} = [{\assign x \query(\qexpr)}]^{l} 
      \\
      \qvar(c_1) \cup \qvar(c_2)  
      & {c} = {c_1};{c_2}
      \\
      \qvar(c_1) \cup \qvar(c_2) 
      & {c} =\eif([\bexpr]^{l}, c_1, c_2) 
      \\
      \qvar(c')
      & {c}   = \ewhile ([\bexpr]^{l}, {c}')
\end{array}
\right.
$$
}
\end{defn}
%
% \subsection{Event and Trace}
% \label{subsec:event_trace}

%
\clearpage
