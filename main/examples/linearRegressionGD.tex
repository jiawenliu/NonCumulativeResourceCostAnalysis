%
\begin{example}[Linear Regression Algorithm with Gradient Decent Optimization]
\label{ex:linearregression}
    The linear regression algorithm with gradient decent Optimization works well 
    in our $\THESYSTEM$ as well.
            %   \[
            %   %
            %   \begin{array}{l}
            %   \kw{linearRegression(step, rate)} \triangleq \\
            %          \clabel{ a \leftarrow 0}^{0} ; \\
            %          \clabel{ c \leftarrow 0}^{1} ; \\
            %           \clabel{\assign{j}{\kw{step}} }^{2} ; \\
            %         %   \clabel{\assign{d}{10000000} }^{2} ; \\
            %           \ewhile ~ \clabel{j > 0}^{3} ~ \edo ~ \\
            %           \Big(
            %               \clabel{\assign{da}{\query(-2 * (\chi[1] - (\chi[0]\times a + c)) \times (\chi[0]))} }^{4}  ; \\
            %               \clabel{\assign{dc}{\query(-2 * (\chi[1] - (\chi[0]\times a + c)))} }^{5}  ; \\
            %               \clabel{\assign{a}{a - \kw{rate} * da} }^{6}  ; \\
            %               \clabel{\assign{c}{c - \kw{rate} * dc} }^{7}  ; \\
            %            \clabel{\assign{j}{j-1}}^{8} 
            %         %   \clabel{a \leftarrow x :: a}^{6} 
            %           \Big);
            %       \end{array}
            %   \]
              %
              %
                   %
\begin{figure}
\centering
\begin{subfigure}{0.45\textwidth}
    \centering
    {\small
        \[
        \begin{array}{l}
            \kw{linearRegressionGD(k, rate)} \triangleq \\
                   \clabel{ a \leftarrow 0}^{0} ; 
                   \clabel{ c \leftarrow 0}^{1} ; 
                    \clabel{\assign{j}{\kw{k}} }^{2} ; \\
                  %   \clabel{\assign{d}{10000000} }^{2} ; \\
                    \ewhile ~ \clabel{j > 0}^{3} ~ \edo ~ \\
                    \Big(
                        \clabel{\assign{da}{\query(-2 * (\chi[1] - (\chi[0]\times a + c)) \times (\chi[0]))} }^{4}  ; \\
                        \clabel{\assign{dc}{\query(-2 * (\chi[1] - (\chi[0]\times a + c)))} }^{5}  ; \\
                        \clabel{\assign{a}{a - \kw{rate} * da} }^{6}  ; 
                        \clabel{\assign{c}{c - \kw{rate} * dc} }^{7}  ; \\
                     \clabel{\assign{j}{j-1}}^{8} 
                  %   \clabel{a \leftarrow x :: a}^{6} 
                    \Big);
                \end{array}
        \]
        }
     \caption{}
        \end{subfigure}
      \begin{subfigure}{.45\textwidth}
          \begin{centering}
          \begin{tikzpicture}[scale=\textwidth/20cm,samples=200]
    % Variables Initialization
    \draw[] (-6, 1) circle (0pt) node{{ $a^0: {}^1_{0}$}};
    \draw[] (-6, 4) circle (0pt) node{{ $c^1: {}^{1}_{0}$}};
    % Variables Inside the Loop
       \draw[] (0, 10) circle (0pt) node{{ $da^4: {}^{k}_{1}$}};
       \draw[] (0, 7) circle (0pt) node{{ $dc^5: {}^{k}_{0}$}};
       \draw[] (0, 4) circle (0pt) node{{ $a^6: {}^{k}_{0}$}};
       \draw[] (0, 1) circle (0pt) node{{ $c^7: {}^{k}_{0}$}};
       % Counter Variables
       \draw[] (7, 9) circle (0pt) node {{$j^0: {}^{1}_{0}$}};
       \draw[] (7, 6) circle (0pt) node {{ $j^8: {}^{k}_{0}$}};
       %
       % Value Dependency Edges:
       \draw[ thick, -latex,] (0, 1.5)  -- (0, 3.5) ;
       \draw[ thick, -Straight Barb] (1.8, 4.2) arc (220:-100:1);
       \draw[ thick, -Straight Barb] (7.5, 6.5) arc (150:-150:1);
       \draw[](10, 6) node[] {\highlight{$k$}} ;
       \draw[ thick, -Straight Barb] (1.7, 1.) arc (120:-200:1);
       \draw[](4, 0) node[] {\highlight{$k$}} ;
       \draw[ thick, -latex] (6, 6.5)  -- 
       node [right] {\highlight{$k$}}(6, 8.5) ;
       % Value Dependency Edges on Initial Values:
       \draw[ thick, -latex,] (-2, 1)  -- 
       node [above] {\highlight{$k$}}(-4.5, 1) ;
       \draw[ thick, -latex,] (-2, 4)  -- 
       node [above] {\highlight{$k$}}(-4.5, 4) ;
       %
       \draw[ ultra thick, -latex, densely dotted,] (-1, 1.5)  to  [out=-220,in=220]  
       node [below] {\highlight{$k$}}(-1, 6.5);
       \draw[ ultra thick, -latex, densely dotted,] (-1, 4.5)  to  [out=-220,in=220]  
       node [above] {\highlight{$k$}}(-1, 9.5);
       \draw[ ultra thick, -latex, densely dotted,]  (1, 6.2) to  [out=-60,in=60] 
       node [below] {\highlight{$k$}}(0.5, 1.5) ;
       \draw[ ultra thick, -latex, densely dotted,]  (1.2, 9.2)  to  [out=-50,in=50] 
       node [above] {\highlight{$k$}}(0.5, 4.5);
       % Control Dependency
      %  \draw[ thick,-latex] (1.5, 7)  -- (4, 9) ;
      %  \draw[ thick,-latex] (1.5, 4)  -- (4, 9) ;
       \draw[ thick,-latex] (1.8, 10)  -- 
       node [above] {\highlight{$k$}}(5.5, 6) ;
       \draw[ thick,-latex] (1.8, 7)  -- (5.5, 6) ;
       \draw[ thick,-latex] (1.8, 4)  -- 
       node [above] {\highlight{$k$}}(5.5, 6) ;
       \draw[ thick,-latex] (1.8, 1)  -- 
       node [below] {\highlight{$k$}}(5.5, 6) ;
       \end{tikzpicture}
       \caption{}
          \end{centering}
          \end{subfigure}
          %
        \begin{subfigure}{.8\textwidth}
            \begin{centering}
            \begin{tikzpicture}[scale=\textwidth/20cm,samples=200]
      % Variables Initialization
      \draw[] (-6, 1) circle (0pt) node{{ $a^0: {}^1_{0}$}};
      \draw[] (-6, 4) circle (0pt) node{{ $c^1: {}^{1}_{0}$}};
      % Variables Inside the Loop
         \draw[] (0, 10) circle (0pt) node{{ $da^4: {}^{k}_{1}$}};
         \draw[] (0, 7) circle (0pt) node{{ $dc^5: {}^{k}_{0}$}};
         \draw[] (0, 4) circle (0pt) node{{ $a^6: {}^{k}_{0}$}};
         \draw[] (0, 1) circle (0pt) node{{ $c^7: {}^{k}_{0}$}};
         % Counter Variables
         \draw[] (7, 9) circle (0pt) node {{$j^0: {}^{1}_{0}$}};
         \draw[] (7, 6) circle (0pt) node {{ $j^8: {}^{k}_{0}$}};
         %
         % Value Dependency Edges:
         \draw[ thick, -latex,] (0, 1.5)  -- (0, 3.5) ;
         \draw[ thick, -Straight Barb] (1.8, 4.2) arc (220:-100:1);
         \draw[ thick, -Straight Barb] (7.5, 6.5) arc (150:-150:1);
         \draw[](10, 6) node[] {\highlight{$\trace_0 \to \env(\trace_0) k $}} ;
         \draw[ thick, -Straight Barb] (1.7, 1.) arc (120:-200:1);
         \draw[](4, 0) node[] {\highlight{$\trace_0 \to \env(\trace_0) k $}} ;
         \draw[ thick, -latex] (6, 6.5)  -- 
         node [right] {\highlight{$\trace_0 \to \env(\trace_0) k $}}(6, 8.5) ;
         % Value Dependency Edges on Initial Values:
         \draw[ thick, -latex,] (-2, 1)  -- 
         node [above] {\highlight{$\trace_0 \to \env(\trace_0) k $}}(-4.5, 1) ;
         \draw[ thick, -latex,] (-2, 4)  -- 
         node [above] {\highlight{$\trace_0 \to \env(\trace_0) k $}}(-4.5, 4) ;
         %
         \draw[ ultra thick, -latex, densely dotted,] (-1, 1.5)  to  [out=-220,in=220]  
         node [below] {\highlight{$\trace_0 \to \env(\trace_0) k $}}(-1, 6.5);
         \draw[ ultra thick, -latex, densely dotted,] (-1, 4.5)  to  [out=-220,in=220]  
         node [above] {\highlight{$\trace_0 \to \env(\trace_0) k $}}(-1, 9.5);
         \draw[ ultra thick, -latex, densely dotted,]  (1, 6.2) to  [out=-60,in=60] 
         node [below] {\highlight{$\trace_0 \to \env(\trace_0) k $}}(0.5, 1.5) ;
         \draw[ ultra thick, -latex, densely dotted,]  (1.2, 9.2)  to  [out=-50,in=50] 
         node [above] {\highlight{$\trace_0 \to \env(\trace_0) k $}}(0.5, 4.5);
         % Control Dependency
        %  \draw[ thick,-latex] (1.5, 7)  -- (4, 9) ;
        %  \draw[ thick,-latex] (1.5, 4)  -- (4, 9) ;
         \draw[ thick,-latex] (1.8, 10)  -- 
         node [above] {\highlight{$\trace_0 \to \env(\trace_0) k $}}(5.5, 6) ;
         \draw[ thick,-latex] (1.8, 7)  -- (5.5, 6) ;
         \draw[ thick,-latex] (1.8, 4)  -- 
         node [above] {\highlight{$\trace_0 \to \env(\trace_0) k $}}(5.5, 6) ;
         \draw[ thick,-latex] (1.8, 1)  -- 
         node [below] {\highlight{$\trace_0 \to \env(\trace_0) k $}}(5.5, 6) ;
         \end{tikzpicture}
         \caption{}
            \end{centering}
            \end{subfigure}
    \vspace{-0.5cm}
    \caption{(a) The linear regression algorithm 
    (b) The program-based dependency graph from $\THESYSTEM$
    (c) The execution-based dependency graph.}
    \vspace{-0.5cm}
    \label{fig:linear_regression}
\end{figure}
%
Analysis Result: $ \progA(\kw{linearRegressionGD(k, rate)}) = k$
\end{example} 
%
 
This linear regression algorithm 
% in order to
aims to
model a linear relationship between a dependent variable $y$,
% corresponding to the observed value in the column $\chi[1]$ in database, 
and an independent variable $x$, $y = a \times x + c$, specifically approximating the 
model parameter $a$ and $c$.
In order to have a good approximation on the model parameter 
$a$ and $c$, 
% corresponding to the observed value in the column $\chi[0]$ in database, 
it sends query to a training data set adaptively in every iteration.
This training data set contains two columns (can extend to higher dimensional data sets), first column is used as the observed value for the independent variable $x$,
second column is used as the observed label value for the dependent variable $y$.
This algorithm is written in our {\tt Query While} language in Figure~\ref{fig:linear_regression}(a) as $\kw{linearRegressionGD(k, rate)}$.
% taking the iteration number $\kw{step}$ 

This linear regression algorithm starts from initializing the linear model parameters and the counter variable,
and then goes into the training iterations.
In each iteration, it computes the differential value w.r.t. parameter
$a$ and $c$ respectively,
through requesting two queries, $\query(-2 * (\chi[1] - (\chi[0]\times a + c)) \times (\chi[0]))$ and 
$\query(-2 * (\chi[1] - (\chi[0]\times a + c)))$
at line 4 and 5.
Then, it uses these two differential values stored in variable $da$ and $dc$ to update the linear model parameters $a$ and $c$.
%
Its the program-based dependency graph is shown in Figure~\ref{fig:linear_regression}(b). Its execution-based dependency graph share the same graph, only needs to change the weight, $k$ into $w_k$ and $1$ for $w_1$ as we do in the previous example.
% We omit the detail of how to 
% generate this graph, which is similar to the generation procedure in 
% Example~\ref{alg:multiRound}.
In the execution-based dependency graph, there are multiple walks having the same longest query length.
For example, the walk $c^7 \to dc^6 : \to c^7 \to \cdots \to dc^6$ along the 
dotted arrows, where each vertex is visited $w_k(\trace_0)$ times for an initial trace $\trace_0$.
% By counting the total occurrence time of vertices with annotation $1$ in this walk, we have this program's adaptivity $k$.
There is actually other walks having the same query length $k$, the 
walk $a^7 \to da^6  \to a^7 \to \cdots \to da^6 $ along the 
dotted arrows, where each vertex is visited $w_k(\trace_0)$ times.
% the dotted path corresponds to a finite walk with the longest query length and its adaptivity on this walk is $k$.
But it doesn't affect the adaptivity for this program, which is still the maximal query length $w_k(\trace_0)$ with respect to initial trace $\trace_0$.
Also, $\THESYSTEM$, estimates the adaptivity $k$ for this example. Similarly as the multiple round example, we can show it is a tight bound.
%